\chapter{Motivation} % (fold)
\label{cha:motivation}

Over the years, great advances have been made in the fields of search and information retrieval. A lot of attention has been brought over the topics of full-text search or document indexing. This great deal of effort has brought upon us tools, such as Google, Yahoo, that allow us to retrieve any document available on the web that might be related to a topic we are searching for. This has had great repercussions upon the world, democratizing and easing the access to information all over the world. However, as we depend more and more upon this type of simple but wide search, we have reached the limits of such a system. It is not documents we are now looking for, but precise answers to questions. It is no longer sufficient to give the user a list of documents that he has to read in order to possibly find an answer. We now need to give the user a proper and actionable answer.

Slow progress has been made in that field, and several projects have implemented some question answering applied to one restricted field or situation, where factual information is available and the input domain is restricted enough to narrow the question down to one possible answer. What is now needed are improvement in two areas: first, the single-domain restriction must be lifted, as question might span more than one domain(e.g. Who was the president of italy the year the man first went to the moon); secondly, a next-generation search engine should be able to provide results for not only questions about precise facts, but also about what we can name judgement calls, questions where many options can be narrowed down and offered according to a precise ranking (e.g. Where can I find a cheap Chinese restaurant in Milan).

The proliferation of open and accessible web services has allowed the world to access, aggregate and mix data in previously unthought ways. This has led a team at Politecnico di Milano to undertake the effort of building a system that pushes the boundaries presented in the previous paragraph. This European Union-funded project is named Search Computing (or SeCo) and while it is currently under active development, it promises to further the field of search in many aspects by providing a holistic aspect to finding answers and data on the Web.

In particular, while many discoveries has been made on the front of the theoretical and formal as well as engineering aspects of distributing a multi-domain query over a series of services, merging back the results in a consistent and interesting manner, there is still a great need for exploration in the subject of interfacing these advances with the user in the most natural and organic fashion possible. Traditionally, such interfaces consist either of a application where the parameters need to be set manually and exhaustively, or a formal syntax that is both strict and cumbersome. Services such as Google or Yahoo! having popularized and democratized the simple input box where free text can be entered, the users now expect more liberty and for the computer to have the burden of understanding the meaning of their sentence. This is what lead us to the current project of research in the field of query analysis, specifically oriented towards understand and translating the queries made in the SeCo project and that can span more than one domain.

Getting back to the original vision of the problem, we see that the project aims to help users answer questions, questions which were previously tedious to search for, both because they required access to opaque services, as well as a lot of manual work to coordinate the answers from the various services in a reasonable manner. Given that we want now to answer these questions for the user, it makes sense to take as input the question itself as it has been formulated by a user, and give the best answer given the system's access to external services which can provide parts of the answers.

In this fashion, query analysis is required to translate the question from the natural way a user would input it, into a form the system can understand and act upon. Most primordially, it is necessary to use query analysis to understand which domains are put into play within the question, leading the system to choose the corresponding services.

\section{Originality of this project} % (fold)
\label{sec:originality}

While much progress has been done in the field of expert or question-answering systems, they mostly limit themselves to either a single topic or a very simple, unnatural syntax. This simplifies the problem greatly, but it still moves the burden of getting interesting results from the system, where it should be, to the user, who now has to put great effort in getting the system to understand him.

Furthermore, existing projects usually take a bottom-up approach, where the data is organized and a service is offered to access it in diverse fashions, and where the input possibilities are dictated by the shape and organization of the data model. On the other hand, ours is following a top-down methodology, taking as base the input of the user, and try to extract meaning from it. This is both required by the great variety and variability that is found in the interacting services, as well as desirable from a point of view, as it will provide a much easier access to complex queries, queries that would have taken pages of declarative queries using previous approaches.

% section originality (end)

\section{Overview} % (fold)
\label{sec:overview}

This project has thus been defined as the search for an algorithm that will allows us to take the natural language input from a user within the context of a multi-domain search engine, and extract the domains from the queries. In order to do this, a corpus of data will be used to test different approaches. Our approach thus relies heavily on external sources of data that will be filtered and chosen for our specific needs.

Once the data is put in place, we will implement and evaluate different algorithms that will try to find a way to split a sentence into many different parts, each one being in a different domain, as well as algorithms that will help us find out which domain is exactly at the core of the parts of sentence previously defined.

% section overview (end)


% chapter motivation (end)