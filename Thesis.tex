\documentclass[a4paper]{report}
\usepackage{fontspec}

\defaultfontfeatures{Mapping=tex-text} % converts LaTeX specials (``quotes'' --- dashes etc.) to unicode
\setromanfont [Ligatures={Common}, Numbers={OldStyle}]{Georgia}
\setmonofont[Scale=0.8]{Monaco} 
\setsansfont[Scale=0.9]{Helvetica Neue} 

% ---- CUSTOM AMPERSAND
\newcommand{\amper}{{\fontspec[Scale=.95]{Hoefler Text}\selectfont\itshape\&}}

\usepackage{graphicx}

\usepackage{sectsty} 
\usepackage[normalem]{ulem}
\chapterfont{\sffamily\bfseries\huge}
\sectionfont{\mdseries\large} 
\subsectionfont{\rmfamily\mdseries\scshape\normalsize} 
\subsubsectionfont{\rmfamily\bfseries\upshape\normalsize}

\usepackage{setspace}

\begin{document}

\onehalfspacing

\title{A Natural Language Analysis Framework for Multi-Domain Queries}
\author{Jean-Philippe bougie}
\date{July 2009}

%\maketitle
\begin{titlepage}
  \setlength{\parindent}{0cm}
  \fontspec[Scale=.95]{Futura}\selectfont
  \begin{flushleft}
    \LARGE{Politecnico di Milano}\\[0.5cm]
    \includegraphics[width=0.15\textwidth]{images/logopolimi}\\[0.5cm]
    \large{Polo Regionale di Como}\\[2cm]
  
    \LARGE{Master of Science in Computer Engineering}\\[2cm]
  
    \Huge{A Natural Language Analysis Framework for Multi-Domain Queries}\\[2cm]
  
    \begin{large}
      \emph{Supervisor:} Professor Marco Brambilla \\
      \emph{A Master Graduation Thesis by}\ Jean-Philippe BOUGIE \\
      \emph{Student Id.} 722963\\[2.5cm]
    \end{large}
  \end{flushleft}
  
  \begin{flushright}
    Academic Year 2008/2009
  \end{flushright}
\end{titlepage}

\pagenumbering{roman}
\tableofcontents
\listoffigures
\listoftables

\chapter*{Acknowledgements}
\begin{flushright}
  \begin{emph}
    Thanks to my family and friends who supported me through this enduring ordeal.\\[1cm]
    Special thanks to Alessandro Bozzon who patiently guided and advised me during the creation of this work.
  \end{emph}
\end{flushright}

\begin{abstract}
  Web Services now allow access to a multitude of data over the Internet. They can now be used conjointly in order to create remarkable results. This goal of allowing automatic resolution of multi-domain queries has been the main research center of the Search Computing project of Politecnico di Milano. While this presents many challenges about the formal aspects of data querying and retrieval, it also leads to an open question as how to present this in an accessible interface.
  
  This interface is thus the subject of this research, where we explore the subject of natural language analysis within the context of multi-domain queries, a subject that has not been touched before in a formal manner.
  We first proceeded by manually building and evaluating a corpus of data from external sources. This corpus was then used to implement and evaluate many algorithms for parsing sentences and extracting the domains which can be used to call the web services. These two main tasks were accomplished through the development of custom tools for the acquisition and rating of data.
  
  The work accomplished presents a good basis for the further development of the field, first for the creation of the corpus which will allow further research on the same topic. The developed tools will also be applicable in similar research or could easily be expanded to suit a more general setting.
  
  The algorithms presented and evaluated also present fruitful ways of splitting the data, and work well in the shown cases. They further present a solid groundwork for the elaboration on an engineering-based solution.
\end{abstract}

\pagenumbering{arabic}


\chapter{Motivation} % (fold)
\label{cha:motivation}

Over the years, great advances have been made in the fields of search and information retrieval. A lot of attention has been brought over the topics of full-text search or document indexing. This great deal of effort has brought upon us tools, such as Google, Yahoo, that allow us to retrieve any document available on the web that might be related to a topic we are searching for. This has had great repercussions upon the world, democratizing and easing the access to information all over the world. However, as we depend more and more upon this type of simple but wide search, we have reached the limits of such a system. It is not documents we are now looking for, but precise answers to questions. It is no longer sufficient to give the user a list of documents that he has to read in order to possibly find an answer. We now need to give the user a proper and actionable answer.

Slow progress has been made in that field, and several projects have implemented some question answering applied to one restricted field or situation, where factual information is available and the input domain is restricted enough to narrow the question down to one possible answer. What is now needed are improvement in two areas: first, the single-domain restriction must be lifted, as question might span more than one domain(e.g. Who was the president of italy the year the man first went to the moon); secondly, a next-generation search engine should be able to provide results for not only questions about precise facts, but also about what we can name judgement calls, questions where many options can be narrowed down and offered according to a precise ranking (e.g. Where can I find a cheap Chinese restaurant in Milan).

The proliferation of open and accessible web services has allowed the world to access, aggregate and mix data in previously unthought ways. This has led a team at Politecnico di Milano to undertake the effort of building a system that pushes the boundaries presented in the previous paragraph. This system called SeCo is under active development and shall be described further in this document.

In particular, while much progress has been made on the front of the theoretical and formal as well as engineering aspects of distributing a multi-domain query over a series of services, merging back the results in a consistent and interesting manner, there is still a great need for exploration in the subject of interfacing these advances with the user in the most natural and organic fashion possible. Traditionally, such interfaces consist either of a application where the parameters need to be set manually and exhaustively, or a formal syntax that is both strict and cumbersome. Services such as Google or Yahoo! having popularized and democratized the simple input box where free text can be entered, the users now expect more liberty and for the computer to have the burden of understanding the meaning of their sentence. This is what lead us to the current project of research in the field of query analysis, specifically oriented towards understand and translating the queries made in the SeCo project and that can span more than one domain.

% chapter motivation (end)

%!TEX root = /Users/jp/Thesis/Thesis.tex
\chapter{Background} % (fold)
\label{cha:background}

\section{Definitions} % (fold)
\label{sec:definitions}

Before getting into deeper details about the project, it is important to clearly define some terms that will be used throughout this report.

\subsection{Web Service} % (fold)
\label{sub:web_service_def}

We think of Web Services in an abstract sense, as providers of content in response to a request, over the Internet. We assume that they can answer a precise query based on a set of defined conditions, and that they will return an ordered set of answers, the order being either based on a criteria provided by the user or by a relevancy score internal to the service. We can further take in consideration that the service has some performance rating, depending on the delay in which it returns the results as well as the quantity and quality of the results. Furthermore, we expect most web services to support some kind of pagination protocol in order to retrieve subsets of the data sequentially. Another important aspect of the services is that they might be subject to a monetary cost, or a limit in the querying rate, based on a daily period or otherwise.

For our purpose, the technology used for the transport and querying interface doesn't matter, be it SOAP, REST, Protobuffers or another binary or text encoding protocol. We assume that an adapter will be developed for each type of Web Service in order to use it.

% subsection web_service_def (end)

\subsection{Domain} % (fold)
\label{sub:domain_def}

A domain is a field of research as provided by one or more search engines. It could be, for example, international events, health professionals, countries information. In a query made by a user, a domain is characterized by the following parts:

\begin{itemize}

  \item One or more objects: While a domain represent an abstract entity, these are the concrete apparitions of the domain within the query. They may correspond directly to the name of the domain (eg. Give me a warm \emph{country}\ with cheap flights on the weekend of the first of may) or not (eg. Give me a \emph{conference}\ about computer science where I can also go to the opera, where conference is an object of the domain ``Event''). Additionally, more than one objects might refer to the same domain, in which case they need to be taken together.

  \item Conditions: Some conditions are given explicitly (eg. Give me a conference \emph{about computer science}.), while others can be deducted from the objects (\emph{conference}\ being a type of event). In general, conditions restrict the field of search that will be sent to the service. Some conditions are objective while others make use of subjective judgements (eg. \emph{warm}\ country).

  \item Ordering: Making the ranking and ordering according to some relevance score is an integral part of this project, and thus the ordering must be extracted and taken into account while answering a user's query. Again, some conditions can be subjective (the \emph{best}\ doctor in the region) while other elements can be taken both as a condition and as an implicit ordering ( \emph{cheap}\ flights will expose a cutoff in addition to the ordering by ascending price).

  \item One or more joins with other domains: at the core of this project is the presence of multiple domains within one query. These different domains must be joined in some way. These joins can be explicit, or inferred through a chain of relationships (an event is an city, that city is in a country) as expressed through a semantic network.

\end{itemize}


% subsection domain_def (end)

% section definitions (end)


\section{SeCo -- Search Computing} % (fold)
\label{sec:seco}

SeCo is the platform which aims to push the limits of the field of multi-domain queries by formalizing theoretical aspects as well as providing a software engineering point of view, enabling the construction of a usable search engine that will answer arbitrary queries.

\subsection{General Architecture} % (fold)
\label{sub:general_architecture}

SeCo is divided in different higher-level components orchestrated in a service-oriented manner. The main components are the query analysis, the query-to-domain mapper, the query planner, the query engine and the results transformation. Two frameworks named the service and domain frameworks are also added as intelligent repositories.

A query sent from the user first passes through the query analysis and the query-to-domain mapper, where the different domains and properties are extracted from the natural language query. It then goes to the query planner, which creates an execution plan taking into accounts the different costs associated to executing the query, in order to create the most efficient execution. The different subqueries are then sent to the domain and service frameworks, which take care of calling the external services through a Web or messaging interface. The results are then collected, and, according to the plan, merged back together. The final results are then transformed before being sent back to the user.

% subsection general_architecture (end)

\subsection{Query Analysis} % (fold)
\label{sub:query_analysis}

In order for a user to query the database of information and knowledge available in SeCo, he or she first needs to input it, whether by a Web interface, or by making a direct call to the programming interface of the system. The type of query the user gives can vary a lot, and it is for the system to try and understand the meaning of the question. In general, the contract of this component is to take the input as given by the user, try to separate it according to the sub-queries that can be found and send those down the line to the query planner.

% subsection query_analysis (end)

\subsection{Query to Domain Mapper} % (fold)
\label{sub:query_to_domain_mapper}

While the set of sub-queries is a first step towards actually executing the query, it still represents an abstract, human request that might or might not correspond to the the services available in the system. The Query to Domain Mapper will take each sub-query in turn, and corresponding to the vocabulary used and the elements in the query, will try to associate a domain as stored in the domain framework. It will send this augmented request to the next component, the query planner.

% subsection query_to_domain_mapper (end)

\subsection{Query Planner} % (fold)
\label{sub:query_planner}

The Query Planner's task is to take the set of high-level queries to different domains and turn it into an executable plan. This involves planning the fetching of data to web services, taking into account the frequency and quantity of data that will be required, and the merge operations that will be done, all in order to minimize the time and memory spent, as well as any costs related to the user of the web services.

This plan is now ready to be executed by the query engine as it contains precise instructions on what to execute and in which order. It is comparable to having taken a declarative program as input and obtaining an imperative program that can be run directly.

% subsection query_planner (end)

\subsection{Query Engine} % (fold)
\label{sub:query_engine}
This component takes the low-level plan from the query planner and executes the different service calls in parallel, merging and ordering when required. It will return the final results of the query in a pure and internal format as they become available, sending them to the results transformation component for their final processing.

Care must be taken in accord to the availability and performance of the invoked services. Since there is an heavy reliance on external components, failures or low performance problems will be common and will have to be considered while fetching the results.
% subsection query_engine (end)

\subsection{Results Transformation} % (fold)
\label{sub:results_transformation}
This component's role is to take the results as given by the Query Engine, and to transform and format them in the way that was required by the client. It implements features such as paging, XML or XHTML output.
% subsection results_transformation (end)

% section seco (end)

\section{Sources of input data} % (fold)
\label{sec:sources_of_input_data}

In order to 
% section sources_of_input_data (end)

\section{Query Analysis} % (fold)
\label{sec:query_analysis}

\subsection{Natural Language Processing} % (fold)
\label{sub:natural_language_processing}

% subsection natural_language_processing (end)

\section{Parse Trees} % (fold)
\label{sec:parse_trees}

% section parse_trees (end)

% section query_analysis (end)

\section{Use of Query Analysis within SeCo} % (fold)
\label{sec:use_of_query_analysis_within_seco}

% section use_of_query_analysis_within_seco (end)

\section{Originality} % (fold)
\label{sec:originality}

% section originality (end)

% chapter background (end)


%!TEX root = /Users/jp/Thesis/Thesis.tex
\chapter{The Natural Language Query Analysis and Evaluation Process} % (fold)
\label{cha:specifications}

Given the originality of the present project, research had to be done in order to find an optimal approach to extracting domains from a typical user input. In this chapter, we will explain the methodology employed to find out such an approach, as well as the tools that were created to reach this purpose.

\section{Methodology} % (fold)
\label{sec:methodology}

To get interesting results from this experiment, a formal methodology was needed. While it is quite simple, it provides us with great flexibility and with constant progress towards the goal of extracting domains from a simple request. This project has two major phases, which shall be described further in details and of which an overview in available in figure~\ref{fig:methodology}. In the first phase, a corpus that responds to the needs of the project, that is, assembled of as many multi-domain queries as possible, has to created from scratch using publicly available data, as well as some manually-entered ones.

In a second phase, from this larger corpus, a smaller but most interesting section is taken and analyzed in depth. Again this analysis has two aspects. The first one is to find a technique that will allow us to split a question in the diverse domains that constitute it, and extract the important objects from those parts. The second one is to take the resulting objects, and associate them with one or more semantical domain of knowledge that will be mapped to the corresponding services. For both these problems, various approaches will be developed and evaluated, individually and relatively to each other.

\begin{figure}[ht!]
  \begin{center}
    \includegraphics[width=\linewidth]{images/methodology}
  \end{center}
  \caption{High-level Methodology}\label{fig:methodology}
\end{figure}

% section methodology (end)

\section{Creation of the corpus} % (fold)
\label{sec:creation_of_the_corpus}

\subsection{Data Extraction} % (fold)
\label{sub:data_extraction}

The first step towards achieving our goal is to obtain an interesting and sizable corpus of data to analyze and on which we can try different algorithms. In order to do this, we used the previously explained choice of Yahoo! Answers, querying it to slowly build a collection of typical user entries.

% subsection data_extraction (end)

\subsection{Entry filtering \amper\ rating} % (fold)
\label{sub:entry_filtering_rating}

Of course, these questions were entered by human beings for the benefit of other human beings, and while they have to be clear enough to be understood by others, no concern has been put in making them clear enough for a computer system to understand them. Thus, a lot of the entries obtained in the previous phase are lacking in content, in clarity or contain too many grammatical errors to consider them as valid entries. They have to be filtered out, or at least flagged as of little importance to our research.

Similarly, we focus our attention on the multi-domain questions, and this is this aspect that we want to characterize in the corpus. We must thus highlight the entries that present this quality. The procedure we use to do that is to give them a rating that go from one, where an entry is useless or of no importance, to five, where an entry represents an excellent subject for research.

This manual rating also presents an excellent opportunity for successive phases in which we can try to use the existing data as training data to try to see if we can automatically sort new entries in the good classification.

% subsection entry_filtering_rating (end)

% section creation_of_the_corpus (end)

\section{Query Analysis} % (fold)
\label{sec:query_analysis_spec}

\subsection{Parsing} % (fold)
\label{sub:parsing}

% subsection parsing (end)

\subsection{Extraction strategies} % (fold)
\label{sub:extraction_strategies}

% subsection extraction_strategies (end)

\subsection{Domain Mapping Strategies} % (fold)
\label{sub:domain_mapping_strategies}

% subsection domain_mapping_strategies (end)

\subsection{Results evaluation} % (fold)
\label{sub:results_evaluation}

% subsection results_evaluation (end)

% section query_analysis_spec (end)

\section{Data Models} % (fold)
\label{sec:data_models}
Different data models are used in the components, both internally and externally. The ones interesting for this research correspond to the ones that will be given as inputs and outputs to the different components and that will allow us to interact with those components.

The Query Analysis component will require two data models; The high-level query is its main input, while it outputs principally an abstract low-level query. The Query-to-domain Mapper will take that abstract query, and replace it with a concrete low-level query. The Domain Framework will make use of the additional data model that is the Domain Descriptor.

\subsection{High-level Query} % (fold)
\label{sub:high_level_query}

A high-level query is given as input to the query analysis, and is for the most part the direct input of the user. It will be given in the form of a quasi-natural question. It can optionally be augmented with some hints that will be in the form of Google Search's keywords, such as "ranking: temperature" or "domain: conference". Internally, this high-level query will be taken as a single string of arbitrary length.

% subsection high_level_query (end)

\subsection{Low-level Query} % (fold)
\label{sub:low_level_query}

A low-level query, either in the abstract or concrete form, will be given in a datalog-like format, that is, as a set of logical predicates describing the properties of diverse objects. These predicates will allow the use of constants, which will define restrictions on the properties of the desired outputs, as well as variables, which will roughly be in a similar amount to the number of different domains that can be extracted from the query. Extensions to the format include complex boolean logic, as well as special predicates that will control the ranking.  Following is an example low-level query, where an uppercase symbol denote a variable and a lowercase one a constant. The associated high-level query for this would be "Give me a conference about computer science in a country where the average temperature is higher than 20 degrees".

\begin{verbatim}
  type(X, conference).
  type(Y, country).
  topic(X, computer science).
  country(X, Y).
  average_temperature(Y, > 20).
\end{verbatim}

Additionally, such a query can be made concrete by annotating it with the corresponding domains, as well as the services that will be needed during the execution of the query, making it into the following example.

\begin{verbatim}
  type(X, conference).
  type(Y, country).
  topic(X, computer science).
  country(X, Y).
  average_temperature(Y, > 20).
  
  domain(X, events).
  domain(Y, country).
  service(X, service_ref(123)).
  service(Y, service_ref(456)).
\end{verbatim}

It is important to note the difference between the type of the object and its domain. In our case, we can see that the Y object share the same type and domain. On the other hand, the X object, of type conference, will be deducted to be an element of the query domain of events. The object in that case becomes a predicate that will restrict the field of search on this particular generic domain.

% subsection low_level_query (end)

\subsection{Domain} % (fold)
\label{sub:domain}

For the purposes of our research, a domain will be simply considered as a mapping between a unique name (eg. Event) and a series of keywords organized either as a simple bag, as an ontology or a taxonomy. It will also contain the different services that can be used to answer a query that corresponds to the selected domain.
% subsection domain (end)

% section data_models (end)

\section{Interfaces} % (fold)
\label{sec:interfaces}

While the components are organized in a chain corresponding to the flow of information in the system, they will be mostly independent from one another, that is, the components will not call directly their successor to continue the execution. Instead, they will be organized in a service fashion, where they will only respond to a request by sending back a response to the client. The client in the main case will be an orchestrator that will take into account each component, calling them in turn and validating the output after each step, taking care of error and exception checking, as well as interaction with the end user through an HTTP or graphical interface.

Thus, each component will be defined individually in regards to its external interface, although one component might make use of another component internally (for example, the Query-to-Domain Mapper will use the Domain Framework).

\subsection{Query Analysis} % (fold)
\label{sub:query_analysis_dm}

The query analysis will respond to a very simple interface. It consists of one method that takes a high-level query as given by a user and outputs a series of abstract low-level queries as presented earlier. These low-level queries will be wrapped in a result object that allow the component to give additional information such as the things it ignored or could not understand, the errors it encountered, which might be useful to help the user refine its search.

\begin{verbatim}
  type result = { annotations: string[], queries: lowLevelQuery[]}
  analyzeQuery(highLevelQuery: string): result
\end{verbatim}

% subsection query_analysis_dm (end)

\subsection{Query to Domain Mapper} % (fold)
\label{sub:query_to_domain_mapper_int}

The query to domain mapper will also follow a simple contract, that is, of taking the abstract low-level queries given directly or transformed from a high-level query through the query analysis, and making them concrete, that is associating the domain and the web service to call to each individual low-level query.

\begin{verbatim}
  mapDomain(queries: lowLevelQuery[]): lowLevelQuery[]
\end{verbatim}

% subsection query_to_domain_mapper_int (end)

\subsection{Domain Framework} % (fold)
\label{sub:domain_framework}

For the purpose of this thesis, the only interfaces that will be needed to use the domain framework will be the ones that allow us to associate a domain from a query object, as well as the one that will help us find a service from a pre-selected domain. To obtain the domain, we will want to give the framework the object we consider at the center of the domain query (eg. \emph{conference}\ in ``give me a conference about computer science...''), as well as the sentence fragment to help the framework disambiguate on certain cases. It will return the most likely domain associated with that object, or None if it can't find anything that could be used.

\begin{verbatim}
  type domain = int // a domain is identified by an integer
  findDomain(object: string, fragment: string): domain
\end{verbatim}

The second interface takes the domain and the queries and will return the best service to use considering the conditions in the query. Some services will take different parameters, and so, depending on the conditions that are requested in the query, a different service might be called. This is why we include the query in the search for a proper service.

\begin{verbatim}
  type service = int // a service is also identified by an integer
  getServiceFromDomain(aDomain: domain, query: lowLevelQuery[]): service
\end{verbatim}

% subsection domain_framework (end)

% section interfaces (end)


% chapter specifications (end)

\chapter{Implementation} % (fold)
\label{cha:implementation}

% chapter implementation (end)

\chapter{Results} % (fold)
\label{cha:results}

\section{Creation of the corpus} % (fold)
\label{sec:creation_of_the_corpus_results}

% section creation_of_the_corpus_results (end)


% chapter results (end)

%!TEX root = /Users/jp/Thesis/Thesis.tex
\chapter{Conclusion} % (fold)
\label{cha:conclusion}

What characterizes the most this project has been the multi-facetted and constructive approach taken, which we can explain by the need for the elaboration of a corpus of data that satisfies our need, and the assembly of a set of tools that power the evaluation of the results. In fact, while interesting results have been uncovered and progress has been made in the search for analysis in a multi-domain query, what leaves the strongest impression are the tools that will act as a foundation for future work, allowing the next research to progress efficiently in the same direction, and also to develop new research that follow the same framework.

What we presented in this work is thus not only a set of data that we can base ourselves on, or an algorithm that splits queries in their diverse domains, but also the whole environment in which the present results were built.

The building blocks, as we might call them, were, on one hand, a Web application that allows easy and continuous extraction and evaluation of external data sets. The tool is simple but highly effective and allows for an efficient use, which is a requirement when one is to process a great amount of data. It also proves flexible and can be quickly adapted for other situations.

The second tool that was built is a focused tool for offloading long or costly processes off to another machine where it can be run asynchronously. While other solutions exist for that purpose, they usually are usually highly complex. This one, evolved from the basic needs, proves simple to use, and, as has been shown throughout this project, is simple to write in and caters to diverse needs.

\section{Future Work} % (fold)
\label{sec:future_work}

If progress has been made in the field, the work is still far from being completed. Apart from an engineering-oriented approach to the solution of domain splitting, there are many opportunities to enhance the results and provide new ways to extract the domains from queries.

\subsection{Optimization} % (fold)
\label{sub:optimization}

What we have given throughout this project is an purified, quite abstract form of the algorithms that could be used as a means of extracting domains from a multi-domain query. While they produce interesting results, many edge cases were not considered, and many improvements to the effectiveness of the program could be made by analyzing a bigger corpus of data.

As an example, some learning strategies could be used for the software to evolve over time by remembering about previous searches from the users. Corrections to commonly mistaken words could be done, and the parser could be tuned by training it on the real data.

% subsection optimization (end)

\subsection{Typed Dependencies} % (fold)
\label{sub:typed_dependencies}

While the core tool provided by the Stanford Parser is the parse tree that we have seen and used throughout this project, other tools are at our disposition to use. One such tool is called the Typed Dependencies representations. Instead of returning a tree, this parser returns a series of typed connections between the words in the sentence. The type of the connection depend on the relationship between the elements, so in the part \emph{the article}, we would have a relationship of the type \emph{determiner} between \emph{the} and \emph{article}. Done over a whole sentence, this creates an annotated graph.

We could then analyze this graph and see its properties in order to find out if its shape bears any relation to the domains of the query. We could do this both in a competitive mode with the tree variant, or cooperatively, using the sum of the two bases of knowledge about the sentence to get a better heuristic.

% subsection typed_dependencies (end)

\subsection{Feedback from the domain framework} % (fold)
\label{sub:feedback_from_the_domain_framework}

In the context of this research, we considered the Query Analysis as a completely independent component from the infrastructure of SeCo. But in reality, it will be one of many different parts that are allowed to communicate. One particularly interesting component will be the domain framework, which contains a base of knowledge about the domains it can query. Within that base, it is planned  that it keeps a set of words related to that domain, or, in some cases an complete ontology. We could thus use that ontology in order to direct the division and the extraction of the domains from the queries

% subsection feedback_from_the_domain_framework (end)

% section future_work (end)

% chapter conclusion (end)

\bibliographystyle{plain}
\bibliography{thesis}

\end{document}
