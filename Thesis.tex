\documentclass[a4paper]{report}
\usepackage{fontspec}

\defaultfontfeatures{Mapping=tex-text} % converts LaTeX specials (``quotes'' --- dashes etc.) to unicode
\setromanfont [Ligatures={Common}, Numbers={OldStyle}]{Georgia}
\setmonofont[Scale=0.8]{Monaco} 
\setsansfont[Scale=0.9]{Helvetica Neue} 

% ---- CUSTOM AMPERSAND
\newcommand{\amper}{{\fontspec[Scale=.95]{Hoefler Text}\selectfont\itshape\&}}

\usepackage{graphicx}

\usepackage{sectsty} 
\usepackage[normalem]{ulem}
\chapterfont{\sffamily\bfseries\huge}
\sectionfont{\mdseries\large} 
\subsectionfont{\rmfamily\mdseries\scshape\normalsize} 
\subsubsectionfont{\rmfamily\bfseries\upshape\normalsize}

\usepackage{setspace}

\begin{document}

\onehalfspacing

\title{A Natural Language Analysis Framework for Multi-Domain Queries}
\author{Jean-Philippe bougie}
\date{July 2009}

%\maketitle
\begin{titlepage}
  \setlength{\parindent}{0cm}
  \fontspec[Scale=.95]{Futura}\selectfont
  \begin{flushleft}
    \LARGE{Politecnico di Milano}\\[0.5cm]
    \includegraphics[width=0.15\textwidth]{images/logopolimi}\\[0.5cm]
    \large{Polo Regionale di Como}\\[2cm]
  
    \LARGE{Master of Science in Computer Engineering}\\[2cm]
  
    \Huge{A Natural Language Analysis Framework for Multi-Domain Queries}\\[2cm]
  
    \begin{large}
      \emph{Supervisor:} Professor Marco Brambilla \\
      \emph{A Master Graduation Thesis by}\ Jean-Philippe BOUGIE \\
      \emph{Student Id.} 722963\\[2.5cm]
    \end{large}
  \end{flushleft}
  
  \begin{flushright}
    Academic Year 2008/2009
  \end{flushright}
\end{titlepage}

\pagenumbering{roman}
\tableofcontents
\listoffigures
\listoftables

\chapter*{Acknowledgements}
\begin{flushright}
  \begin{emph}
    Thanks to my family and friends who supported me through this enduring ordeal.\\[1cm]
    Special thanks to Alessandro Bozzon who patiently guided and advised me during the creation of this work.
  \end{emph}
\end{flushright}

\begin{abstract}
  Web Services now allow access to a multitude of data over the Internet. They can now be used conjointly in order to create remarkable results. This goal of allowing automatic resolution of multi-domain queries has been the main research center of the Search Computing project of Politecnico di Milano. While this presents many challenges about the formal aspects of data querying and retrieval, it also leads to an open question as how to present this in an accessible interface.
  
  This interface is thus the subject of this research, where we explore the subject of natural language analysis within the context of multi-domain queries, a subject that has not been touched before in a formal manner.
  We first proceeded by manually building and evaluating a corpus of data from external sources. This corpus was then used to implement and evaluate many algorithms for parsing sentences and extracting the domains which can be used to call the web services. These two main tasks were accomplished through the development of custom tools for the acquisition and rating of data.
  
  The work accomplished presents a good basis for the further development of the field, first for the creation of the corpus which will allow further research on the same topic. The developed tools will also be applicable in similar research or could easily be expanded to suit a more general setting.
  
  The algorithms presented and evaluated also present fruitful ways of splitting the data, and work well in the shown cases. They further present a solid groundwork for the elaboration on an engineering-based solution.
\end{abstract}

\pagenumbering{arabic}


\chapter{Motivation} % (fold)
\label{cha:motivation}

Over the years, great advances have been made in the fields of search and information retrieval. A lot of attention has been brought over the topics of full-text search or document indexing. This great deal of effort has brought upon us tools, such as Google, Yahoo, that allow us to retrieve any document available on the web that might be related to a topic we are searching for. This has had great repercussions upon the world, democratizing and easing the access to information all over the world. However, as we depend more and more upon this type of simple but wide search, we have reached the limits of such a system. It is not documents we are now looking for, but precise answers to questions. It is no longer sufficient to give the user a list of documents that he has to read in order to possibly find an answer. We now need to give the user a proper and actionable answer.

Slow progress has been made in that field, and several projects have implemented some question answering applied to one restricted field or situation, where factual information is available and the input domain is restricted enough to narrow the question down to one possible answer. What is now needed are improvement in two areas: first, the single-domain restriction must be lifted, as question might span more than one domain(e.g. Who was the president of italy the year the man first went to the moon); secondly, a next-generation search engine should be able to provide results for not only questions about precise facts, but also about what we can name judgement calls, questions where many options can be narrowed down and offered according to a precise ranking (e.g. Where can I find a cheap Chinese restaurant in Milan).

The proliferation of open and accessible web services has allowed the world to access, aggregate and mix data in previously unthought ways. This has led a team at Politecnico di Milano to undertake the effort of building a system that pushes the boundaries presented in the previous paragraph. This system called SeCo is under active development and shall be described further in this document.

In particular, while much progress has been made on the front of the theoretical and formal as well as engineering aspects of distributing a multi-domain query over a series of services, merging back the results in a consistent and interesting manner, there is still a great need for exploration in the subject of interfacing these advances with the user in the most natural and organic fashion possible. Traditionally, such interfaces consist either of a application where the parameters need to be set manually and exhaustively, or a formal syntax that is both strict and cumbersome. Services such as Google or Yahoo! having popularized and democratized the simple input box where free text can be entered, the users now expect more liberty and for the computer to have the burden of understanding the meaning of their sentence. This is what lead us to the current project of research in the field of query analysis, specifically oriented towards understand and translating the queries made in the SeCo project and that can span more than one domain.

% chapter motivation (end)

%!TEX root = /Users/jp/Thesis/Thesis.tex
\chapter{Background} % (fold)
\label{cha:background}

\section{Introduction}
Over the years, great advances have been made in the fields of search and information retrieval. A lot of attention has been brought over the topics of full-text search or document indexing. This great deal of effort has brought upon us tools, such as Google, Yahoo, that allow us to retrieve any document available on the web that might be related to a topic we are searching for. This has had great repercussions upon the world, democratizing and easing the access to information all over the world. However, as we depend more and more upon this type of simple but wide search, we have reached the limits of such a system. It is not documents we are now looking for, but precise answers to questions. It is no longer sufficient to give the user a list of documents that he has to read in order to possibly find an answer. We now need to give the user a proper and actionable answer.

Slow progress has been made in that field, and several projects have implemented some question answering applied to one restricted field or situation, where factual information is available and the input domain is restricted enough to narrow the question down to one possible answer. What is now needed are improvement in two areas: first, the single-domain restriction must be lifted, as question might span more than one domain(e.g. Who was the president of italy the year the man first went to the moon); secondly, a next-generation search engine should be able to provide results for not only questions about precise facts, but also about what we can name judgement calls, questions where many options can be narrowed down and offered according to a precise ranking (e.g. Where can I find a cheap Chinese restaurant in Milan).

The proliferation of open and accessible web services has allowed the world to access, aggregate and mix data in previously unthought ways. This has led a team at Politecnico di Milano to undertake the effort of building a system that pushes the boundaries presented in the previous paragraph. This system called SeCo is under active development and shall be described further.


\section{SeCo -- Search Computing}

SeCo is a platform that aims to enable the type of advanced research described before.

\subsection{General Architecture}

SeCo is divided in different higher-level components orchestrated in a service-oriented manner. The main components are the query analysis, the query-to-domain mapper, the query planner and the query engine. Two frameworks named the service and domain frameworks are also added as intelligent repositories.

A query sent from the user first passes through the query analysis and the query-to-domain mapper, where the different domains and properties are extracted from the natural language query. It then goes to the query planner, which creates an execution plan taking into accounts the different costs associated to executing the query, in order to create the most efficient execution. The different subqueries are then sent to the domain and service frameworks, which take care of calling the external services through a Web or messaging interface. The results are then collected, and, according to the plan, merged back together. The final results are then sent back to the user.

\subsection{Query Analysis}

In order for a user to query the database of information and knowledge available in SeCo, he or she first needs to input it, whether by a Web interface, or by making a direct call to the programming interface of the system. The type of query the user gives can vary a lot, and it is for the system to try and understand the meaning of the question. In general, the contract of this component is to take the input as given by the user, try to separate it according to the sub-queries that can be found and send those down the line to the query planner.

\subsection{Query to Domain Mapper} % (fold)
\label{sub:query_to_domain_mapper}

% subsection query_to_domain_mapper (end)
% chapter background (end)


%!TEX root = /Users/jp/Thesis/Thesis.tex
\chapter{The Natural Language Query Analysis and Evaluation Process} % (fold)
\label{cha:specifications}

Given the originality of the present project, research had to be done in order to find an optimal approach to extracting domains from a typical user input. In this chapter, we will explain the methodology employed to find out such an approach, as well as the tools that were created to reach this purpose.

\section{Methodology} % (fold)
\label{sec:methodology}

To get interesting results from this experiment, a formal methodology was needed. While it is quite simple, it provides us with great flexibility and with constant progress towards the goal of extracting domains from a simple request. This project has two major phases, which shall be described further in details and of which an overview in available in figure~\ref{fig:methodology}. In the first phase, a corpus that responds to the needs of the project, that is, assembled of as many multi-domain queries as possible, has to created from scratch using publicly available data, as well as some manually-entered ones.

In a second phase, from this larger corpus, a smaller but most interesting section is taken and analyzed in depth. Again this analysis has two aspects. The first one is to find a technique that will allow us to split a question in the diverse domains that constitute it, and extract the important objects from those parts. The second one is to take the resulting objects, and associate them with one or more semantical domain of knowledge that will be mapped to the corresponding services. For both these problems, various approaches will be developed and evaluated, individually and relatively to each other.

\begin{figure}[ht!]
  \begin{center}
    \includegraphics[width=\linewidth]{images/methodology}
  \end{center}
  \caption{High-level Methodology}\label{fig:methodology}
\end{figure}

% section methodology (end)

\section{Creation of the corpus} % (fold)
\label{sec:creation_of_the_corpus}

\subsection{Data Extraction} % (fold)
\label{sub:data_extraction}

The first step towards achieving our goal is to obtain an interesting and sizable corpus of data to analyze and on which we can try different algorithms. In order to do this, we chose to use Yahoo! Answers, as it offers many benefits. First, it consists of a extensive number of questions, out of which we can find many which present the features of multi-domain queries. In addition, the easy and open access to the Application Programming Interface insures that we can acquire data without relying on fickle screen scraping techniques that are prone to break whenever there is a slight change. The API instead provides a stable gateway into their data for those who wish to respectfully make use of it in research or for new applications.

The Yahoo! Answers API has a few entry points, but we are only interested in one, which, given a category name or a category unique identifier, returns a set of questions that were registered in that category by the users. Each question contains a set of metadata that includes the unique identifier of the question, the url at which we can see it on Yahoo!'s web site, the title, a description, the date and time at which it was added and the number of answers to that question. Out of these, we keep track only of the unique identifier and the title of the question, which usually presents a small but interesting view of it.

% subsection data_extraction (end)

\subsection{Basic Filtering} % (fold)
\label{sub:basic_filtering}

Out of these entries, we can proceed to a first level of filtering where we remove any entry that are clearly too off base. Our criteria for this is to remove any entry where no question is actually asked, or where the words contain grammatical mistakes in all the words that are the core of the sentence, that is, the nouns and verbs that define the question, as our technique currently relies on a dictionary trained on published and edited texts. In further phases, pre-processing in the form of a spell check could be envisaged, but it is of no use for the current phase of building an interesting and consistent corpus of data.

As the text of the question is divided into two spaces, that is the title and the description, some users have preferred to insert a generic phrase into the title and instead give all the details in the description. Since the corpus we are trying to build must represent typical entries into a query engine under the form of a single-line input box, we only wish to keep the title which has the same form. Thus, any entry where the title has no question must be torn off from the repository.

As a pre-processing phase, the data will be slightly sanitized, for examples by putting the first letter in the upper case form, as well as removing any excessive punctuation

% subsection basic_filtering (end)

\subsection{Manual Evaluation \amper\ Rating} % (fold)
\label{sub:entry_filtering_rating}

Of course, these questions were entered by human beings for the benefit of other human beings, and while they have to be clear enough to be understood by others, no concern has been put in making them clear enough for a computer system to understand them. Thus, a lot of the entries obtained in the previous phase are still lacking in content, in clarity.

Most importantly, we focus our attention on the multi-domain questions, and this is this core aspect of what we want to characterize in the corpus. We must thus highlight the entries that present this quality. The procedure we use to do that is to give them a rating that go from one, where an entry is useless or of no importance, to five, where an entry represents an excellent subject for research.

% subsection entry_filtering_rating (end)

% section creation_of_the_corpus (end)

\section{Query Analysis} % (fold)
\label{sec:query_analysis_spec}

Once a suitable corpus has been built, we can proceed to the creation of techniques that will allow us to recognize and extract the domains from a question. This procedure will span many different phases, each one of them requiring different steps, from the elaboration of diverse strategies, the application of them on the corpus, and the evaluation of the results. These steps are accomplished in a cyclical manner, until the results are interesting enough to move on to the next phase.

\subsection{Parsing} % (fold)
\label{sub:parsing}

The first step of the process of analysis of the queries is to parse the sentence, that is, to transform it from a simple linear and unadorned presentation into a tree representing the grammatical structure found within the sentence. This annotated structure is what will be used in the subsequent phases to split the sentence into multiple domains.

In order to do this, two open source tools were chosen to be evaluated. The first one is the Stanford Natural Language Parser \footnote{See http://nlp.stanford.edu}, created at Stanford University in the United States. It is a Java-based library, that, accompanied with a trained corpus of data for the English language, will obtain a parse tree with each atom being annotated with its role, and the different structures corresponding to the parts of the sentence (object, verb, complement), being grouped into the tree. This parser uses as its internal as well as external representation, the Penn Treebank form, a widely-used formalism for representing parts of speech developed at the University of Pennsylvania. It also provides a simple Application Programming Interface where the corpus data is loaded, and then the parser is then applied to an entry, giving back the completed parse tree.

To offer a comparison of the first tool, we selected the University of Illinois at Urbana-Champaign tool called the Shallow Parser \footnote{See shallow parser}. This tool, while also based on a trained set of data, presents a shallow view of the parse tree instead of a deep one like the Stanford tool does. It means that it only groups parts of the sentence with a one level precision, whereas the Stanford tool can go infinitely deep, but typically groups things down from 7 to 10 levels deep. Instead of being a library, the Shallow Parser is installed as a series of Unix tools followed by a server that answers to queries. First, the input must be sanitized and prepared for the parsing by spacing the punctuation and by performing many small transformations. The output from this first tool is then fed into a part of speech tagger that will parse the sentence and annotate each word or token with its corresponding part of the speech (noun, verb, adjective, adverb, etc.). The result is then sent to the final server, called the chunking server, that will take this series of tagged tokens and group some of them according to its heuristics, returning the final output.

The two tools were compared on the quality of the results as well as on their general performance, both in terms of processing time and memory usage. The quality of the results can be defined as how accurately they annotate each atom of the sentence, which corresponds to the process of part of speech tagging, and how well they group the atoms into significant parts of the sentence which correspond to the diverse elements of the sentence (subject, action, object).

% subsection parsing (end)

\subsection{Spliting \amper\ Extraction strategies} % (fold)
\label{sub:extraction_strategies}

Following the parsing of the input corpus, the next step is to devise a strategy that will divide the single input into multiple parts of the sentence, where each part roughly corresponds to a single domain that will be searched. This is where the importance of the structure obtained in the first part comes into play, as it is by leveraging that structure and its properties that we will find many opportunities to split a domain into many different parts.

Many different techniques were thought about, but two main techniques were retained and tested on the input. The first one is to split directly at the first level of the sentence, on the assumption that the upper levels of the trees will be divided in different sub-sentences where each one corresponds to a single unique domain.

A second technique is to split directly according to the parser's recognition of clauses, either subjunctive or relative. It is expressed in the tree by the presence of internal nodes that have been labeled with one of the following symbols: S, SQ, SBAR, SBARQ, FRAG (refer to Table~\ref{tab:penntree_symbols} for explanations about the different symbols and their meaning).

While splitting a sentence into different parts is an important step towards getting the final domains out of an entry, it still returns a response that is too coarse. That is, the parts of the sentence still contain far too many data which is of little use for us in the context of extracting the domains. In order to proceed to the next step, we must thus reduce each parts to its simplest objects, and the ones that will characterize the domain in which we find ourselves. Thus, from each part, we keep only the nouns and the verb if they correspond to meaningful actions (e.g \emph{drive}, \emph{rent}, \emph{cure}).

% subsection extraction_strategies (end)

\subsection{Domain Mapping Strategies} % (fold)
\label{sub:domain_mapping_strategies}

From these basic objects, nouns and verbs, we can use another set of tools and techniques to extract a meaningful domain out of it, a domain that can be mapped to a web service later using other components of SeCo. In order to do this, we focus on the tools provided by the WordNet project, and especially the add-on of WordNet Domains.

The technique we use is to parse the dictionary of WordNet, which is organized by words who relate to one or more synonym sets or senses of a word, also called synsets. Each synset has a unique identifier consisting of its offset within the WordNet database. This identifier can be used to connect to its associated domains within the WordNet-Domains database, where the key is the synset offset and its values are one or more domains. Refer to Figure~\ref{fig:wordnethierarchy} to see the relationships we follow to get the domains. As it can be seen, this results in a large number of domains from a single word. In order to get the most relevant domains, we use the tf-idf information retrieval technique \footnote{See http://en.wikipedia.org/wiki/tf-idf} as a sorting mechanism, calculating the importance of a single domain by its relative presence in a single word over how common it is across all the domains we retrieved from the objects of the part.

A second technique we use is to retrieve the \emph{topic} relationship directly from WordNet, which gives a word definition a relationship to another word of which it is the topic. In order to do this, we need to go into the WordNet database and extract the interesting entries, following the offset links as they come.

\begin{figure}[ht!]
  \begin{center}
    \includegraphics[width=\linewidth]{images/wordnethierarchy}
  \end{center}
  \caption{WordNet Hierarchy}\label{fig:wordnethierarchy}
\end{figure}

% subsection domain_mapping_strategies (end)

% section query_analysis_spec (end)

\section{Data Models} % (fold)
\label{sec:data_models}
Different data models are used in the components, both internally and externally. The ones interesting for this research correspond to the ones that will be given as inputs and outputs to the different components and that will allow us to interact with those components.

The Query Analysis component will require two data models; The high-level query is its main input, while it outputs principally an abstract low-level query. The Query-to-domain Mapper will take that abstract query, and replace it with a concrete low-level query. The Domain Framework will make use of the additional data model that is the Domain Descriptor.

\subsection{High-level Query} % (fold)
\label{sub:high_level_query}

A high-level query is given as input to the query analysis, and is for the most part the direct input of the user. It will be given in the form of a quasi-natural question. It can optionally be augmented with some hints that will be in the form of Google Search's keywords, such as "ranking: temperature" or "domain: conference". Internally, this high-level query will be taken as a single string of arbitrary length.

% subsection high_level_query (end)

\subsection{Low-level Query} % (fold)
\label{sub:low_level_query}

A low-level query, either in the abstract or concrete form, will be given in a datalog-like format, that is, as a set of logical predicates describing the properties of diverse objects. These predicates will allow the use of constants, which will define restrictions on the properties of the desired outputs, as well as variables, which will roughly be in a similar amount to the number of different domains that can be extracted from the query. Extensions to the format include complex boolean logic, as well as special predicates that will control the ranking.  Following is an example low-level query, where an uppercase symbol denote a variable and a lowercase one a constant. The associated high-level query for this would be "Give me a conference about computer science in a country where the average temperature is higher than 20 degrees".

\begin{verbatim}
  type(X, conference).
  type(Y, country).
  topic(X, computer science).
  country(X, Y).
  average_temperature(Y, > 20).
\end{verbatim}

Additionally, such a query can be made concrete by annotating it with the corresponding domains, as well as the services that will be needed during the execution of the query, making it into the following example.

\begin{verbatim}
  type(X, conference).
  type(Y, country).
  topic(X, computer science).
  country(X, Y).
  average_temperature(Y, > 20).
  
  domain(X, events).
  domain(Y, country).
  service(X, service_ref(123)).
  service(Y, service_ref(456)).
\end{verbatim}

It is important to note the difference between the type of the object and its domain. In our case, we can see that the Y object share the same type and domain. On the other hand, the X object, of type conference, will be deducted to be an element of the query domain of events. The object in that case becomes a predicate that will restrict the field of search on this particular generic domain.

% subsection low_level_query (end)

\subsection{Domain} % (fold)
\label{sub:domain}

For the purposes of our research, a domain will be simply considered as a mapping between a unique name (eg. Event) and a series of keywords organized either as a simple bag, as an ontology or a taxonomy. It will also contain the different services that can be used to answer a query that corresponds to the selected domain.
% subsection domain (end)

% section data_models (end)

\section{Interfaces} % (fold)
\label{sec:interfaces}

While the components are organized in a chain corresponding to the flow of information in the system, they will be mostly independent from one another, that is, the components will not call directly their successor to continue the execution. Instead, they will be organized in a service fashion, where they will only respond to a request by sending back a response to the client. The client in the main case will be an orchestrator that will take into account each component, calling them in turn and validating the output after each step, taking care of error and exception checking, as well as interaction with the end user through an HTTP or graphical interface.

Thus, each component will be defined individually in regards to its external interface, although one component might make use of another component internally (for example, the Query-to-Domain Mapper will use the Domain Framework).

\subsection{Query Analysis} % (fold)
\label{sub:query_analysis_dm}

The query analysis will respond to a very simple interface. It consists of one method that takes a high-level query as given by a user and outputs a series of abstract low-level queries as presented earlier. These low-level queries will be wrapped in a result object that allow the component to give additional information such as the things it ignored or could not understand, the errors it encountered, which might be useful to help the user refine its search.

\begin{verbatim}
  type result = { annotations: string[], queries: lowLevelQuery[]}
  analyzeQuery(highLevelQuery: string): result
\end{verbatim}

% subsection query_analysis_dm (end)

\subsection{Query to Domain Mapper} % (fold)
\label{sub:query_to_domain_mapper_int}

The query to domain mapper will also follow a simple contract, that is, of taking the abstract low-level queries given directly or transformed from a high-level query through the query analysis, and making them concrete, that is associating the domain and the web service to call to each individual low-level query.

\begin{verbatim}
  mapDomain(queries: lowLevelQuery[]): lowLevelQuery[]
\end{verbatim}

% subsection query_to_domain_mapper_int (end)

\subsection{Domain Framework} % (fold)
\label{sub:domain_framework}

For the purpose of this thesis, the only interfaces that will be needed to use the domain framework will be the ones that allow us to associate a domain from a query object, as well as the one that will help us find a service from a pre-selected domain. To obtain the domain, we will want to give the framework the object we consider at the center of the domain query (eg. \emph{conference}\ in ``give me a conference about computer science...''), as well as the sentence fragment to help the framework disambiguate on certain cases. It will return the most likely domain associated with that object, or None if it can't find anything that could be used.

\begin{verbatim}
  type domain = int // a domain is identified by an integer
  findDomain(object: string, fragment: string): domain
\end{verbatim}

The second interface takes the domain and the queries and will return the best service to use considering the conditions in the query. Some services will take different parameters, and so, depending on the conditions that are requested in the query, a different service might be called. This is why we include the query in the search for a proper service.

\begin{verbatim}
  type service = int // a service is also identified by an integer
  getServiceFromDomain(aDomain: domain, query: lowLevelQuery[]): service
\end{verbatim}

% subsection domain_framework (end)

% section interfaces (end)

% chapter specifications (end)

\chapter{Implementation} % (fold)
\label{cha:implementation}

\begin{figure}[ht!]
  \begin{center}
    \includegraphics[width=\linewidth]{images/architecture}
  \end{center}
  \caption{The High-level Architecture}\label{fig:architecture}
\end{figure}


\section{Data Gathering \amper\ Analysis} % (fold)
\label{sec:data_gathering_and_analysis}

In order to grasp the intricacies of the problem of parsing natural language queries in our particular domain, we tried to look for a source of a large number of already-made queries that correspond to our criteria, that is that they span multiple domains and that they are expressed naturally. We found Yahoo! Answers to be a good source for such kind of data, albeit with a very low signal to noise ratio. That is, a first step on the way to obtaining a valuable corpus would be to extract and filter the data from this site. Thankfully, Yahoo! Answers provide an Application Programming Interface that allows us to obtain questions in large quantities in a XML or JSON format. We thus created a Web Tool, code-named Siphon, that extracts the questions from Yahoo! Answers and present them to the user where he or she can decide whether to accept or reject each entry.

\begin{huge}
  Insert Screenshot
\end{huge}

This tool presents a simple list view where, by default, the yet unfiltered entries are shown. Through the use of keyboard shortcuts or the provided links, the user can circle through the entries, accepting or rejecting them. The entries are stored permanently, whether accepted or not, and it is then possible to change mode and go look at the previously accepted or rejected entries, before coming back to filter new ones. The tool will load new entries from Yahoo! Answers as the number of unfiltered entries get low, insuring a steady amount of entries to filter.

Another interesting functionality of this tool is the ability to inspect individual entries, giving a view of how they would be parsed using both the shallow parser and the Stanford parser, which shall be described further. Parsing a large corpus such as this one is not a trivial matter, and so, a second tool was developed to offload and distribute the parsing tasks to diverse computers. This tool, called Bee, uses a simple queue service as a central dispatching mechanism and connects to a document database to store the results. Its concept is centered around tasks, defined by the developer, which are run by workers waiting for messages on a pre-defined queue. In answer to these parameterized messages, the worker will launch the task and store its result on the database. On the other side, we have the Siphon tool posting messages to this queue as new entries are created, and using the results from the document database to show to the user.

The queue used is Kestrel, which is a simple open-source queue implementation in Scala reusing the now common Memcache protocol, and which has proved its stability and performance at Twitter Inc. where it was developed. As we could expect the schema of the documents to change often as new tasks are added to the service, a schema-less database was chosen. Given the breadth of technology used, the Apache CouchDB project was used as it internally keeps data in the JSON format which has a wide support across different languages. The data structures available in JSON, like objects and arrays, were an essential asset for the transcoding of the parse trees to a common format. The Bee software itself is also written in Scala, providing an easy object-oriented concurrency model based on the actors concept.

In the spirit of open source and given the strong use of it during the development of this project, the two tools described here are made available as free and open source software and can be found at www.github.com/jpbougie.

% section data_gathering_and_analysis (end)

\section{Parsers} % (fold)
\label{sec:parsers}

The first phase of the query analysis was to extract the grammatical structure of the question asked by the user. The existing approaches that are currently widely used are based either on formal grammars, or on keyword extraction.

The first approach is to require the user to input a very structured and often unnatural query. The best example that can be given is that of SQL, the Structured Query Language, that specifies a declarative programming language used by the client as the only mean to access the data. Queries formally define the kind of data that is to be retrieved, the 


 In order to do that, it was decided to use a natural language parser, that can turn a simple sentence into an annotated series of tokens organized in a hierarchical manner according to the grammatical structures and features of the sentence. Two such tools were chosen to be evaluated, the Stanford Natural Language Parser and the University of Illinois at Urbana-Champaign's Shallow Parser. 

While they both use different techniques and thus produce different results for the same input, they have great similarities. Both are based on a statistical and probabilistic analysis that will discern the most likely structure from the tokens in the sentence, assigning each one a role in the sentence as well a grouping the structures together.

\subsection{Stanford Parser} % (fold)
\label{sub:stanford_parser}
The Stanford Parser is a Java library created at Stanford University that is able to turn natural language input into structured trees annotated with the grammatical roles of each part of the sentence and the structure that links those tokens together. Here is an example of what the stanford parser would output for a simple input:
% subsection stanford_parser (end)
% section parsers (end)

% chapter implementation (end)

\chapter{Results} % (fold)
\label{cha:results}

\section{Creation of the corpus} % (fold)
\label{sec:creation_of_the_corpus_results}

\subsection{Quantities} % (fold)
\label{sub:quantities}

% subsection quantities (end)

\subsection{Repartition of the ratings} % (fold)
\label{sub:repartition_of_the_ratings}

% subsection repartition_of_the_ratings (end)

\subsection{Interesting Elements} % (fold)
\label{sub:interesting_elements}

% subsection interesting_elements (end)

% section creation_of_the_corpus_results (end)

\section{Extraction of the domain} % (fold)
\label{sec:extraction_of_the_domain}

\subsection{Choice of the parser} % (fold)
\label{sub:choice_of_the_parser}

We first began by evaluating the two proposed parsers, the Stanford Parser and the Shallow Parser, over the set of accumulated data. We evaluated them over the quality of the results as well as the performance in terms of CPU usage and memory.

Given the relative simplicity of the results given by the Shallow Parser, one might expect it to be much easier to calculate, and be much simpler to use. In fact, what we have found out is that in general, parsing an element using the shallow parser takes as much as three times longer than doing the same with the Stanford parser. This can be explained by the fact that this parser requires a series of tool to be called sequentially, loosing a lot of time in the overhead of passing the elements from one program to another. On the other hand, the stanford is a unique Java library where everything is loaded from the start, before any requests are actually made.

The results given by the Stanford parser were also obviously more finer, as the shallow parser only does a coarse grouping. It was also less reliable, frequently returning no results because the input was not given in a proper english, a situation where the Stanford parser still managed to make an intelligible parsing.

It was thus very easy for us to choose to continue solely with the Stanford parser, both because of the quality of its results and its performance.

% subsection choice_of_the_parser (end)

% section extraction_of_the_domain (end)

% chapter results (end)

%!TEX root = /Users/jp/Thesis/Thesis.tex
\chapter{Conclusion} % (fold)
\label{cha:conclusion}

What characterizes the most this project has been the multi-facetted and constructive approach taken, which we can explain by the need for the elaboration of a corpus of data that satisfies our need, and the assembly of a set of tools that power the evaluation of the results. In fact, while interesting results have been uncovered and progress has been made in the search for analysis in a multi-domain query, what leaves the strongest impression are the tools that will act as a foundation for future work, allowing the next research to progress efficiently in the same direction, and also to develop new research that follow the same framework.

What we presented in this work is thus not only a set of data that we can base ourselves on, or an algorithm that splits queries in their diverse domains, but also the whole environment in which the present results were built.

The building blocks, as we might call them, were, on one hand, a Web application that allows easy and continuous extraction and evaluation of external data sets. The tool is simple but highly effective and allows for an efficient use, which is a requirement when one is to process a great amount of data. It also proves flexible and can be quickly adapted for other situations.

The second tool that was built is a focused tool for offloading long or costly processes off to another machine where it can be run asynchronously. While other solutions exist for that purpose, they usually are usually highly complex. This one, evolved from the basic needs, proves simple to use, and, as has been shown throughout this project, is simple to write in and caters to diverse needs.

\section{Future Work} % (fold)
\label{sec:future_work}

If progress has been made in the field, the work is still far from being completed. Apart from an engineering-oriented approach to the solution of domain splitting, there are many opportunities to enhance the results and provide new ways to extract the domains from queries.

\subsection{Optimization} % (fold)
\label{sub:optimization}

What we have given throughout this project is an purified, quite abstract form of the algorithms that could be used as a means of extracting domains from a multi-domain query. While they produce interesting results, many edge cases were not considered, and many improvements to the effectiveness of the program could be made by analyzing a bigger corpus of data.

As an example, some learning strategies could be used for the software to evolve over time by remembering about previous searches from the users. Corrections to commonly mistaken words could be done, and the parser could be tuned by training it on the real data.

% subsection optimization (end)

\subsection{Typed Dependencies} % (fold)
\label{sub:typed_dependencies}

While the core tool provided by the Stanford Parser is the parse tree that we have seen and used throughout this project, other tools are at our disposition to use. One such tool is called the Typed Dependencies representations. Instead of returning a tree, this parser returns a series of typed connections between the words in the sentence. The type of the connection depend on the relationship between the elements, so in the part \emph{the article}, we would have a relationship of the type \emph{determiner} between \emph{the} and \emph{article}. Done over a whole sentence, this creates an annotated graph.

We could then analyze this graph and see its properties in order to find out if its shape bears any relation to the domains of the query. We could do this both in a competitive mode with the tree variant, or cooperatively, using the sum of the two bases of knowledge about the sentence to get a better heuristic.

% subsection typed_dependencies (end)

\subsection{Feedback from the domain framework} % (fold)
\label{sub:feedback_from_the_domain_framework}

In the context of this research, we considered the Query Analysis as a completely independent component from the infrastructure of SeCo. But in reality, it will be one of many different parts that are allowed to communicate. One particularly interesting component will be the domain framework, which contains a base of knowledge about the domains it can query. Within that base, it is planned  that it keeps a set of words related to that domain, or, in some cases an complete ontology. We could thus use that ontology in order to direct the division and the extraction of the domains from the queries

% subsection feedback_from_the_domain_framework (end)

% section future_work (end)

% chapter conclusion (end)

\bibliographystyle{plain}
\bibliography{thesis}

\end{document}
